\documentclass{article}
\usepackage[a4paper, margin=1in]{geometry}
\usepackage{amsmath, amssymb, graphicx, listings, xcolor}
\usepackage{hyperref}

\title{Programming Assignment: Image Restoration}
\author{Marek Rychlik}
\date{\today}

\lstdefinestyle{pythonstyle}{
    backgroundcolor=\color{gray!5},
    keywordstyle=\color{blue},
    commentstyle=\color{gray},
    stringstyle=\color{magenta},
    basicstyle=\ttfamily\footnotesize,
    breaklines=true,
    numbers=left,
    numberstyle=\tiny\color{gray},
    frame=single,
    captionpos=b,
    showspaces=false,
    showstringspaces=false,
    showtabs=false
}
\lstset{style=pythonstyle}

\begin{document}

\maketitle

\section{Objective}
This assignment explores the the image restoration technique ROF in MATLAB:
\begin{itemize}
\item \textbf{Reading and interpretation of raw image data} – Develop
  understanding about digital images as they are collected by sensors
  (CCD); how they are created, represented and converted to known RGB
  model.

\item \textbf{Non-linear PDE discretization} –
  Developing and implementing an algorithm for solving the discretized
  ROF equation.
  
\item \textbf{Measuring noise in images using ROF} - Analyzing
  statistics of the difference between and image and its smoothed
  version, using ROF.

\end{itemize}

\section{Problem Statement}
The \emph{regularized ROF} is given by the functional:
\[ \mathcal{F}(u) = \int \sqrt{\epsilon^2 + |\nabla u|^2} + \frac{\lambda}{2} \int (u - f)^2 \, dx \, dy \]
%
where \( u \) is the image to be restored, \( f \) is the observed
degraded image, and \( \lambda \) is a regularization parameter.

The Euler-Lagrange PDE derived from this functional is:
\[
\frac{u - f}{\lambda} - \frac{\partial}{\partial x} \frac{u_x}{\sqrt{\epsilon^2 + u_x^2 + u_y^2}}
- \frac{\partial}{\partial x} \frac{u_x}{\sqrt{\epsilon^2 + u_x^2 + u_y^2}}
= 0
\]
%
The Neumann Boundary Conditions on a rectangular domain \( [a, b] \times [c, d] \) are:
\[
\begin{aligned}
  \frac{\partial u}{\partial x}(a, y) &= 0 \quad \text{and} \quad \frac{\partial u}{\partial x}(b, y) = 0, \\
  \frac{\partial u}{\partial y}(x, c) &= 0 \quad \text{and} \quad \frac{\partial u}{\partial y}(x, d) = 0.
\end{aligned}
\]

Develop a difference scheme (by either discretizing the functional \(\mathcal{F}\) or the PDE)
which provides a faithful representation of the ROF technique. Note that
the input data is the discretized fersion of the function \(f\) - the observed degraded image.
The degradation is a result noise during collection of the data, due to the underlying
physics (counting photons, analogue-to-digital conversion, etc.).

The input data is a single image \texttt{DSC00099.ARW}, containing
a \emph{Bayer mosaic}, with RGGB CFA layout. The synopsis of
basic I/O operations on such image data is in the following script:

\begin{lstlisting}[language=MATLAB, caption={Processing of raw images}]
%----------------------------------------------------------------
% File:     basic_script.m
%----------------------------------------------------------------
%
% Author:   Marek Rychlik (rychlik@arizona.edu)
% Date:     Mon Apr  7 14:19:32 2025
% Copying:  (C) Marek Rychlik, 2020. All rights reserved.
% 
%----------------------------------------------------------------
% Basic operations on raw images
% Source image: https://www.reddit.com/r/EditMyRaw/comments/1jt4ecw/the_official_weekly_raw_editing_challenge/
raw_img_filename=fullfile('.','images','DSC00099.ARW')
%raw_img_filename=fullfile('.','images','credit @signatureeditsco - signatureedits.com _DSC4583.dng')
t=tiledlayout('TileSpacing','compact','Padding','compact');
linked_axes=[]
cfa=rawread(raw_img_filename);
ax=nexttile, imagesc(bitand(cfa,  7)),title('xor with 7');
info = rawinfo(raw_img_filename);
disp(info);
Iplanar=raw2planar(cfa);
planes='rggb';
ang = info.ImageSizeInfo.ImageRotation;
for j=1:size(Iplanar,3)
    ax=nexttile; imagesc(imrotate(Iplanar(:,:,j),ang)), title(['Plane ',num2str(j),': ', planes(j)]), colorbar, colormap gray
    linked_axes=[linked_axes,ax];
end
Idemosaic=demosaic(cfa,planes);
% nexttile, image(Idemosaic), colorbar;
Irgb=raw2rgb(raw_img_filename);
% Resized RGB image, so that size matches the size of the planes
ax=nexttile; image(imresize(Irgb,.5)), colorbar,
title('Demosaiced and scaled RGB image');
linked_axes=[linked_axes,ax];
linkaxes(linked_axes);
\end{lstlisting}

Your program will operate separately on the planar data (R,G,G,B) to
compare the differences in the level of noise of different colors.
The entire mosaic is a 2D array, 4024 by 6048. Every plane is half
that size.  Thus, your implementation of ROF will operate on a
``grayscale'' image of size 2014 by 3025. This is the array
\(f_{i,j}\) which approximates \(f\). Note that
\[f_{i,j} = u(x_j, y_i) \qquad \text{Not \(u(x_i,y_j)\)!!!}\]

To evaluate the noise level, you should study the statistics
of the difference \(u-f\) as function of the parameters \((\lambda, \epsilon)\).
The measure of noise will be the \emph{mean square difference}:
\[ MSD(f, \lambda,\epsilon) = \sqrt{\frac{\sum_{i=1}^H\sum_{j=1}^W(u_{i,j}-f_{i,j})^2}{H\cdot W}}\]
where \(H\) and \(W\) are the image height and width, respectively,
and \(u\) is the solution to the ROF problem.

You should graph \(MSD\) over a range that best illustrates the
differences between the level of noise in the planes R, G, G, B.

\section{Tasks}
\subsection{Design and implement the difference scheme for ROF}
\begin{itemize}
\item A MATLAB function \texttt{smooth\_image\_rof} ( implemented in the
  MATLAB function file \texttt{smooth\_image\_rof.m}) with signature
  given below, solving the discretized problem.
\begin{lstlisting}[language=MATLAB, caption={Processing of raw images}]
  u = smooth_image_rof(f, lambda, epsilon)
  % SMOOTH_IMAGE_ROF - perform ROF image restoration
  %  U = SMOOTH(F, LAMBDA, EPSILON) - perform ROF image restoration
  %  Arguments:
  %    F - the degraded image
  %    LAMBDA - the 'smoothing' parameter (scalar or vector)
  %    EPSILON - the 'regularization' parameter  (scalar or vector)
  % Returns:
  %    U - the restored/smoothed image
  % If LAMBDA or EPSILON is a vector, the result should be a 4D array of size
  % H-by-W-by-K-by-L, where H and W are the height and width of F
  % and K and L are the lengths of LAMBDA and EPSILON, respectively.
  
\end{lstlisting}
\item A MATLAB function \texttt{calculate\_msd} (in the file \texttt{calculate\_msd.m})
  which yields the measure of noise in \(f\). The
  signature of the function is:
\begin{lstlisting}[language=MATLAB, caption={Processing of raw images}]
  msd = calculate_msd(f, lambda, epsilon)
  % CALCULATE_MSD - returns MSD for a given degraded image and ROF parameters
  %  MSD = CALCULATE_MSD(F, LAMBDA, EPSILON) - find MSD
  %  Arguments:
  %    F - the degraded image
  %    LAMBDA - the 'smoothing' parameter (scalar or vector)
  %    EPSILON - the 'regularization' parameter (scalar or vector)
  % Returns:
  %    MSD - the MSD of the degraded image (scalar or 1D array)
  % If LAMBDA or EPSILON is a vector, the result should be an array of size
  % K-by-L, where and K and L are the lengths of LAMBDA and EPSILON, respectively.
\end{lstlisting}
You should call \texttt{smooth\_image\_rof} in this function.
\end{itemize}

Note that both functions must be fully vectorized. Thus, if paramegers
\texttt{lambda} and \texttt{epsilon} are vectors,
\texttt{smooth\_image\_rof} must return a family of restored/denoised
images. The result is a 4D array.  Similarly, \texttt{calculate\_msd}
should return a 2D array in this situation, representing the values of
the function $MSD(f, \lambda,\epsilon)$ over the grid
\(\lambda\times\epsilon\). Note that this allows easy plotting using
the \texttt{meshgrid} function of MATLAB. This will be helpful in
constructing the report.

\subsection{Support for GPU and multiple CPU}
Your code should use GPU to accellerate calculations when
available. The function \texttt{gpuArray} fails if there is no GPU
device available on the machine on which the program
runs. Unfortunately, the server on which your code is tested does not
have a compatible GPU (it is an AMD machine with RADEON graphics card;
only NVIDIA graphics cards are supported by MATLAB). Also, you can
call \text{gpuDevice} to test if a GPU is available.

If there is no GPU, your program should try to use multiple CPU. You
can detect the number of computational threads by running
\texttt{maxNumCompThreads}.  On the machine on which your program will
be tested, the number of computational threads is \(4\).  It is a
requirement that your program uses multiple threads.

\section{Deliverables}
\subsection{Code Implementation}
\begin{itemize}
\item MATLAB function files \texttt{smooth\_image\_rof.m}
  and  \texttt{calculate\_msd.m}.
\item Other files that your implementation may depend on.
\end{itemize}

\subsection{Plots and Analysis}
\begin{itemize}
\item For the 4 color planes of the supplied image, R, G, G, B,
  plot the function:
  \[ (\lambda, \epsilon) \mapsto MSD(f, \lambda, \epsilon ) \]
  These should be 4 surfaces stacked one upon another.
  Make sure the surfaces are semi-transparent
  not to obtract each other. Pick a good region of parameters.
\item Make a definite inference about which color planes
  are the most and least noisy.
\item Speculate (in an educated way) why there are
  two green planes in the Bayer mosaic.
\end{itemize}

\subsection{Written Report}
\begin{itemize}
    \item Brief explanation of methods.
    \item Interpretation of results.
    \item Discussion of noise differences between the color planes.
\end{itemize}

\section{High-Performance GPU Batching Strategy}

To improve performance when evaluating the ROF model over a dense grid of parameters
\((\lambda, \epsilon)\), we exploit the parallel processing capabilities of the GPU. 
We batch the parameter sweep by copying the input image into a 4D tensor and process 
all parameter combinations in a single GPU pass.

\subsection{Memory Considerations}

Given an image of size \( H \times W \) (e.g., \(2000 \times 3000\)) and a 
parameter grid of size \( K \times L \) (e.g., \(20 \times 20\)), the required
GPU memory for storing the repeated image across the parameter grid is:
\[
\text{Memory} = H \cdot W \cdot K \cdot L \cdot \text{bytes per pixel}
\]
For a 16-bit image, this is typically \( \sim 12\,\text{MB} \) per copy, 
so 400 copies will require about \(4.7\,\text{GB}\), well within an 8\,GB 
GPU's capacity.

\subsection{Parameter Grid Expansion}

Let \( f \in \mathbb{R}^{H \times W} \) be the degraded image. Define:
\[
f^{(k,l)} = f, \quad \text{for all } (k,l) \in \{1,\ldots,K\} \times \{1,\ldots,L\}
\]
and stack these into a 4D array:
\[
\mathbf{F} \in \mathbb{R}^{H \times W \times K \times L}
\]
The parameters \( \lambda_{k} \), \( \epsilon_{l} \) are similarly broadcast 
to tensors \( \Lambda \) and \( \mathcal{E} \) of shape \(1 \times 1 \times K \times L\).

\subsection{Vectorized ROF Iteration}

For each iteration \( t \), compute:
\begin{align*}
u_x^{(k,l)} &= u^{(k,l)}_{i,j+1} - u^{(k,l)}_{i,j}, \\
u_y^{(k,l)} &= u^{(k,l)}_{i+1,j} - u^{(k,l)}_{i,j}, \\
|\nabla u|^{(k,l)} &= \sqrt{ \epsilon_l^2 + (u_x^{(k,l)})^2 + (u_y^{(k,l)})^2 }, \\
p_x^{(k,l)} &= u_x^{(k,l)} / |\nabla u|^{(k,l)}, \quad
p_y^{(k,l)} = u_y^{(k,l)} / |\nabla u|^{(k,l)}, \\
\text{div}(p)^{(k,l)} &= \text{backward difference of } p_x^{(k,l)} \text{ and } p_y^{(k,l)}, \\
u^{(k,l)}_{t+1} &= f - \lambda_k \cdot \text{div}(p)^{(k,l)}.
\end{align*}

All operations are performed in parallel on the GPU using MATLAB's \texttt{gpuArray}
broadcasting semantics.

\subsection{MSD Computation}

After convergence, compute the mean square difference (MSD) directly on the GPU:
\[
\text{MSD}^{(k,l)} = \sqrt{ \frac{1}{HW} \sum_{i,j} (u_{i,j}^{(k,l)} - f_{i,j})^2 }
\]

\subsection{Benefits}

\begin{itemize}
    \item Reduces kernel launch overhead and memory transfers.
    \item Amortizes the cost of reading and storing \(f\) over many parameter evaluations.
    \item Enables full vectorization of the ROF update loop.
    \item GPU memory bandwidth is used more effectively.
\end{itemize}



\end{document}

%%% Local Variables:
%%% mode: latex
%%% TeX-master: t
%%% TeX-engine: xetex
%%% TeX-command-extra-options: "-shell-escape"
%%% LaTeX-biblatex-use-Biber: t
%%% End:

