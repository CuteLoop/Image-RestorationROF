\documentclass[12pt]{article}
\usepackage[utf8]{inputenc}
\usepackage{geometry}
\usepackage{hyperref}
\geometry{margin=1in}

\title{Frankie Manning and the Shim Sham}
\author{Joel Maldonado}
\date{}

\begin{document}
\maketitle

Every culture has its iconic dances everyone knows—a mixture of choreography and likable songs. In a \emph{quinceañera} everyone will get together and dance to ``Payaso de Rodeo''; in a country bar there will be a song everyone gets excited to line-dance to; in a club someone will be taught how to dougie. In a swing-dance party, anywhere in the world, when the song ``'\,Tain't What You Do (It's the Way That Cha Do It)'' by Jimmie Lunceford and His Orchestra starts playing, everyone will get together to dance along to the choreography we all know as the Shim Sham, the version choreographed by the Ambassador of Lindy himself, Frankie Manning.

Frankie Manning was born in Jacksonville, Florida, but lived in New York since he was three, living with his mom, a dancer (she was fifteen). Frankie’s interest in dancing started when he was eight years old. His mom used to take him to rent parties—parties with a 25-cent cover, food, dancing, and bathtub gin priced at 10 cents a mug. But what captivated young Frankie was seeing the dancing and dancers enjoying themselves. By the time he was ten he would venture into the party and observe, then, going home, he would mimic and copy the moves. This was his first time dancing.

His first time officially partner-dancing was in October after his twelfth birthday when his mom offered to take him to a dance if he helped decorate for Halloween. All night he stood watching dancers until his mom asked young Frankie to dance, at the end receiving the comment, ``You’re too stiff.''

Quite a discouraging comment to start a dance journey. But after reflecting a bit, Frankie took it as constructive criticism and practiced dancing every time he could. ``I would pick a broom or a chair as a partner and try to dance like the grown-ups I had seen, moving my shoulders, my waist, and my hips. I would bend from side to side, doing any little thing just to keep myself from being upright and staring. Sometimes my mom would open the door and find me dancing with a broom. When she asked what I was doing I told her I was trying to get unstiffed.''

He would soon practice partner dancing with his friend \rule{2cm}{0.15mm}, preparing for parties. One Sunday, while going to church with his friend, they discovered there was a live band playing for youngsters aged 12–15. Filled with enthusiasm, they decided to skip church and go dancing—the typical experience when you arrive at the dance and promise to ask a girl to dance but never do. Eventually, on the fourth time, he did.

His first encounter with Lindy was after high school. During a basketball game a player celebrated with a smooth swing move that sparked Frankie’s curiosity. After asking, he was pointed in the direction of the Savoy.

Manning recognizes Lindy as a successor of three dances that were being done in Harlem in the ’20s: the Charleston, the Collegiate, and the Breakaway. Lindy Hop is a partner dance and most dances are solo; in contrast, with partner dancing we connect and get a conversation going. Lindy Hop is such a happy dance and good exercise too. He started dancing Lindy Hop because that was the dance danced everywhere.

His life then revolved around going to the Savoy Ballroom and dancing. He quickly gained recognition and learned from greats like Snowden and Big Bea, who had a move where Bea would have Shorty George jumping back. This triggered one of the iconic moments of competition.

Frankie wins with an aerial landing on beat. In my experience, half the time another dancer talks about Lindy, they talk about their acrobatics.

During the Second World War Frankie served, and afterwards the leaderboards changed into bebop and rock and roll. Social Lindy took a toll. He worked for 30 years in the Postal Service. Yet Lindy Hop never left him. In 1985 a group of young dancers tracking the roots of swing found Frankie, then in his seventies, and convinced him to teach again. Here starts the revival of the swing era. Manning went all over the world traveling and teaching the Lindy.

During this time he adapted the Shim Sham to the version we know today.

\section*{Bibliography}
\begin{enumerate}
  \item ``Breaking Boundaries: The First Air Step in Lindy Hop.'' \emph{SwingDanceHome}.\\
        \url{https://swingdancehome.com/en/breaking-boundaries-the-first-air-step-in-lindy-hop/4}
  \item ``First Aerial in Lindy Hop: Talk by Frankie Manning.'' YouTube.\\
        \url{https://www.youtube.com/watch?v=4ENgHBJMaLk}
  \item ``Frankie Manning.'' YouTube, FLYP Media.\\
        \url{https://www.youtube.com/watch?v=1NKBolRDyBo}
  \item ``Frankie Manning.'' \emph{Wikipedia}.\\
        \url{https://en.wikipedia.org/wiki/Frankie_Manning}
  \item ``Shim Sham for Frankie Manning.'' YouTube.\\
        \url{https://youtu.be/g_k_BIA_unI}
  \item ``Frankie Manning Performs the Shim Sham.'' YouTube, NOLA.com.\\
        \url{https://www.youtube.com/watch?v=KhnNhr1spoM}
  \item Manning, Frankie, and Cynthia R.\ Millman. \emph{Frankie Manning: Ambassador of Lindy Hop}. Temple University Press, 2007.
\end{enumerate}

\end{document}
