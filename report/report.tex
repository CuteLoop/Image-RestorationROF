\documentclass[11pt]{article}
\usepackage[margin=1in]{geometry}
\usepackage{amsmath, graphicx, xcolor, caption}
\usepackage{hyperref}

\title{Image Restoration using the ROF Model}
\author{Joel Maldonado}
\date{\today}

\begin{document}
\maketitle

\section*{Objective}
This report explores image denoising using the Rudin–Osher–Fatemi (ROF) model. We apply the model to color planes of a Bayer-mosaic image and analyze noise levels through Mean Square Difference (MSD) statistics.

\section*{Method}
We solve the ROF model:
\[
\mathcal{F}(u) = \int \sqrt{\epsilon^2 + |\nabla u|^2} + \frac{\lambda}{2} \int (u - f)^2 \, dx\,dy
\]
using a vectorized iterative scheme implemented in MATLAB. The solver adapts to CPU or GPU hardware and computes the solution over a parameter grid $(\lambda, \epsilon)$.

MSD is defined as:
\[
\text{MSD}(f, \lambda, \epsilon) = \sqrt{\frac{1}{HW} \sum_{i,j} (u_{i,j} - f_{i,j})^2}
\]
where \( u \) is the denoised output.

\section*{Results}
The following figure shows MSD surfaces for R, G1, G2, and B planes.

\begin{figure}[h!]
    \centering
    \includegraphics[width=0.95\textwidth]{msd_surfaces.png}
    \caption{MSD surfaces across $(\lambda, \epsilon)$ for each color plane. Offsets were applied to aid visualization.}
\end{figure}

\subsection*{Observations}
\begin{itemize}
  \item Green planes (G1, G2) consistently had lower MSD, indicating less noise.
  \item Blue showed the highest noise, followed by red.
  \item This aligns with the Bayer mosaic design: two green sensors increase spatial luminance resolution.
\end{itemize}

\section*{Discussion}
The ROF model effectively separates noise from signal. Higher values of $\lambda$ enforce more smoothing, while $\epsilon$ stabilizes the gradient flow near flat regions. GPU batching was used to compute MSD over a grid efficiently.

\section*{Conclusion}
ROF is a robust model for image denoising. We confirmed expected noise behavior across channels, and the vectorized approach scales well for parameter exploration.

\end{document}
